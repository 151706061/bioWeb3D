\documentclass{bioinfo}
\usepackage{color}
\usepackage{listings}
\lstset{ %
language=JavaScript                % choose the language of the code
basicstyle=\footnotesize,       % the size of the fonts that are used for the code
numbers=left,                   % where to put the line-numbers
numberstyle=\footnotesize,      % the size of the fonts that are used for the line-numbers
stepnumber=1,                   % the step between two line-numbers. If it is 1 each line will be numbered
numbersep=5pt,                  % how far the line-numbers are from the code
backgroundcolor=\color{white},  % choose the background color. You must add \usepackage{color}
showspaces=false,               % show spaces adding particular underscores
showstringspaces=false,         % underline spaces within strings
showtabs=false,                 % show tabs within strings adding particular underscores
frame=single,           % adds a frame around the code
tabsize=2,          % sets default tabsize to 2 spaces
captionpos=b,           % sets the caption-position to bottom
breaklines=true,        % sets automatic line breaking
breakatwhitespace=false,    % sets if automatic breaks should only happen at whitespace
escapeinside={\%*}{*)}          % if you want to add a comment within your code
}
\copyrightyear{2005}
\pubyear{2005}

\begin{document}
\firstpage{1}

\title[Application Note]{Supplementary for : bioWeb3D: a simple online webGL 3D data visualisation tool}
\author[Pettit \textit{et~al}]{Jean-Baptiste Pettit\,$^{*}$ and John Marioni}
\address{EMBL-EBI, European Molecular Biology Laboratory - European Bioinformatics Institute, Cambridge, CB10 1SD, UK}

\history{Received on XXXXX; revised on XXXXX; accepted on XXXXX}

\editor{Associate Editor: XXXXXXX}

\maketitle
\section{Input file formats}
\subsection{Raw data file}

\vbox{When you add a raw data file, a new Dataset section is going to be created in the "Data" panel of the application. One raw data file contains one dataset. The dataset is composed only of 3D coordinates (x,y,z) along with a few information about the dataset (optional name). Here is what a minimal 3 points dataset file should look like :
\begin{lstlisting}
{ "dataset" : {
      "name" : "my superb dataset",
        "points" : [
          [
            0.5,
            100,
            -50.5
          ],
          [
            200,
            10,
            0.0
          ],
          [
            3,
            250.15,
            15
          ]
        ]
     }
}
\end{lstlisting}}

\subsection{Cluster information files}

The cluster information file contains information about the dataset you have entered with the raw data file. There is no need to repeat the coordinates of the points, and as you can notice in the previous file, points don't have a unique ID. In fact the information you will enter in this file will have to keep the order of the points you defined in the first file. You can have multiple cluster sets in the same file each cluster set has :
\begin{itemize}
\item{a name}
\item{a number of categories called numClust}
\item{A list of labels for the clusters (optional)}
\end{itemize}
For example coming back to the 3 points defined previously, say I have : 
\begin{itemize}

\item{one clustering algorithm that put the first two together in the first cluster and the second one alone in a third cluster.}
\item{another clustering algorithm that put the 3 point in 3 separate clusters the file should look like that}
\end{itemize}
\vbox{
\begin{lstlisting}
{ "cluster" :
  [
    {
      "name": "clustering algo 1",
      "numClust": "2",
      "labels" : [
        "Category 1",
        "Category 2"
      ],
      "values": [
        1,
        1,
        2
      ]
    },
    {
      "name": "clustering algo 2",
      "numClust": "3",
      "values": [
        1,
        2,
        3
      ]
    }
  ]
}
\end{lstlisting}}
\end{document}
